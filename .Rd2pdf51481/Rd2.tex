\documentclass[a4paper]{book}
\usepackage[times,inconsolata,hyper]{Rd}
\usepackage{makeidx}
\usepackage[latin1]{inputenc} % @SET ENCODING@
% \usepackage{graphicx} % @USE GRAPHICX@
\makeindex{}
\begin{document}
\chapter*{}
\begin{center}
{\textbf{\huge \R{} documentation}} \par\bigskip{{\Large of \file{man/ConstrainedGlm.fit.Rd} etc.}}
\par\bigskip{\large \today}
\end{center}
\inputencoding{utf8}
\HeaderA{ConstrainedGlm.fit}{Constrained logistic regression In this function we create a regression for which the predicted probabilities are contstrained. That is, they can not be less than a minimum of delta, or a maxiumum of 1 - delta.}{ConstrainedGlm.fit}
%
\begin{Description}\relax
Constrained logistic regression
In this function we create a regression for which the predicted
probabilities are contstrained. That is, they can not be less than a minimum
of delta, or a maxiumum of 1 - delta.
\end{Description}
%
\begin{Usage}
\begin{verbatim}
ConstrainedGlm.fit(formula, delta, data, fall_back_to_glm = TRUE,
  previous_glm = NULL, ...)
\end{verbatim}
\end{Usage}
%
\begin{Arguments}
\begin{ldescription}
\item[\code{formula}] the formula to use for the regression

\item[\code{delta}] the threshold used for constraining the probabilities

\item[\code{data}] the data to train the glm on

\item[\code{fall\_back\_to\_glm}] boolean should we fall back to traditional glm

\item[\code{previous\_glm}] glm object. A previously trained GLM instance, so it can be updated

\item[\code{...}] the other arguments passed to GLM
\end{ldescription}
\end{Arguments}
%
\begin{Value}
a fitted glm
\end{Value}
\inputencoding{utf8}
\HeaderA{ConstrainedGlm.fit\_new\_glm}{Fit a new GLM In this function we create a new instance a (constrained) glm fit.}{ConstrainedGlm.fit.Rul.new.Rul.glm}
%
\begin{Description}\relax
Fit a new GLM
In this function we create a new instance a (constrained)
glm fit.
\end{Description}
%
\begin{Usage}
\begin{verbatim}
ConstrainedGlm.fit_new_glm(formula, family, data, fall_back_to_glm, ...)
\end{verbatim}
\end{Usage}
%
\begin{Arguments}
\begin{ldescription}
\item[\code{formula}] the formula to use for the regression

\item[\code{family}] the family used for fitting the GLM (binomial, etc)

\item[\code{data}] the data to train the glm on

\item[\code{fall\_back\_to\_glm}] boolean should we fall back to traditional glm

\item[\code{...}] the other arguments passed to GLM
\end{ldescription}
\end{Arguments}
%
\begin{Value}
a fitted glm
\end{Value}
\inputencoding{utf8}
\HeaderA{ConstrainedGlm.predict}{Predict using a constrained glm In this function we predict usng an instance of a (constrained) glm fit.}{ConstrainedGlm.predict}
%
\begin{Description}\relax
Predict using a constrained glm
In this function we predict usng an instance of a (constrained)
glm fit.
\end{Description}
%
\begin{Usage}
\begin{verbatim}
ConstrainedGlm.predict(constrained_glm, newdata)
\end{verbatim}
\end{Usage}
%
\begin{Arguments}
\begin{ldescription}
\item[\code{constrained\_glm}] a fitted constrained glm instance

\item[\code{newdata}] the data to perform the predictions with
\end{ldescription}
\end{Arguments}
%
\begin{Value}
a prediction
\end{Value}
\inputencoding{utf8}
\HeaderA{ConstrainedGlm.update\_glm}{Update Constrained logistic regression In this function we update a previously trained instance of a (constrained) glm fit.}{ConstrainedGlm.update.Rul.glm}
%
\begin{Description}\relax
Update Constrained logistic regression
In this function we update a previously trained instance of a (constrained)
glm fit.
\end{Description}
%
\begin{Usage}
\begin{verbatim}
ConstrainedGlm.update_glm(previous_glm, data, ...)
\end{verbatim}
\end{Usage}
%
\begin{Arguments}
\begin{ldescription}
\item[\code{previous\_glm}] glm object. A previously trained GLM instance, so it can be updated

\item[\code{data}] the newdata to update the glm on

\item[\code{...}] the other arguments passed to GLM
\end{ldescription}
\end{Arguments}
%
\begin{Value}
a fitted, updated glm
\end{Value}
